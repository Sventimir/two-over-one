\documentclass[12pt,a4paper,twoside]{book}
\usepackage[a4paper,margin=2cm]{geometry}
\usepackage{fancyhdr}
\usepackage{style}

\pagestyle{fancy}
\rhead{Baichuan Yu -- Marcin Pastudzki}
\lhead{Two-over-one bidding system description}

\title{Two-over-one bidding system description}
\author{Marcin Pastudzki \and Baichuan Yu}
\date{\today}


\begin{document}

\maketitle
\tableofcontents

\chapter{System overview}

This is the full description of \system\/ as used by pair Marcin Pastudzki -- Baichuan Yu. It might be different from
versions of the same system used by other pairs or described by other sources available.

\system\/ is mostly a \nat\/ bidding system with little conventional improvements in second round of the bidding.
It is based on natural openings according to the scheme: 5-card major, 4-card diamonds and \snat\ \ctr{1}{\c} opening.
No trump openings are \nat, 15-17 and 20-22 respectively.

We use several well known conventions such as:
\begin{itemize}
  \item \emph{mini-Multi} \art\/ opening;
  \item \gf\/ two-over one responses;
  \item \force\/ \ctr{1}{NT} in response to \ctr{1}{\major} opening;
  \item \emph{Bergen raises};
  \item \emph{inverted minors};
  \item \emph{two-way checkback};
  \item \emph{fourth suit} \gf\/;
  \item \emph{Jacoby} \ctr{2}{NT};
  \item \emph{splinter}\/;
  \item \emph{Drury}\/;
  \item \emph{Lebensohl}\/;
\end{itemize}

\end{document}
