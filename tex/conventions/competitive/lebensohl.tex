\subsection{Lebensohl}
\label{subsec:leben}

\conv{Lebensohl} is mainly used in the following sequences:
\begin{itemize}
  \item \sequence{1\nt\ --- (2\major) --- ?}
  \item \sequence{(2\major) --- \dbl\ --- (\pass) --- ?}
\end{itemize}

The basic approach is that we have a special bid of \ctr{2\nt}, which requests \emph{partner} to bid \ctr{3\c}
automatically so that we can make further description. Partner may omit automatic response if he considers himself
having extra values, however, he must be aware of the fact that \ctr{2\nt} \emph{bidder} might only have been preparing
a \so\/ in certain suit and might not be interested in playing another.

After such \emph{opponent's} intervention there are 4 \emph{types} of bids we might want to make and 3 different \emph{intentions}
we might have in mind:

\begin{center}
  \newcommand{\head}[1]{\emph{\underline{#1}}}

  \begin{tabular}{lll}
    \head{type of a bid}    & \hspace{5cm}  & \head{intention} \\

    below opponent's suit   &               & \so     \\
    opponents suit          &               & \inv    \\
    above opponent's suit   &               & \gf     \\
    no trump                &               &         \\
  \end{tabular}
\end{center}

\dbl\/ is always \pen\/ (as is \pass\/ to \emph{partner's} \dbl\/ of course).

Bidding a new suit (or \pass\/ to automatic \ctr{3\c}) declares willingness to play that suit with the intention shown
in the table to the left (X signifies an unbid suit or \pass\/ in case of \ctr{3\c}).

Bidding \emph{opponent's} suit or \nt\ declares willingness to play \nt\ with or without the possibility to play an
unbid major suit, as shown in the table to the right. These are always \gf.

\begin{center}
  \renewcommand{\nt}{NT}
  \renewcommand{\ctr}[1]{#1}
  \newcommand{\leben}[1]{\ctr{2\nt}/\ctr{#1}}
  \newcommand{\direct}[1]{\ctr{#1}}
  \small

  \begin{tabular}{|c|c|c|}
    \hline
          & below               & above               \\
    \hline
    \so   & \ctr{2\nt}/\ctr{3X} & \ctr{2X}            \\
    \inv  & \ctr{3X}            & \ctr{3X}            \\
    \gf   & ---                 & \ctr{2\nt}/\ctr{3X} \\
    \hline
  \end{tabular}
  \quad
  \begin{tabular}{|c|c|c|}
    \hline
                & with \suit{4}{\major}   & without \suit{4}{\major}  \\
    \hline
    stopper    & \leben{3\nt}            & \direct{3\nt}             \\
    no stopper & \leben{3 opp's suit}    & \direct{3 opp's suit}     \\
    \hline
  \end{tabular}

\end{center}
